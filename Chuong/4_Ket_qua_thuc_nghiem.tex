\documentclass[../DoAn.tex]{subfiles}
\begin{document}

\section{Thiết kế kiến trúc hệ thống}

\subsection{Mô hình tổng quan}
Dựa trên yêu cầu về tính thời gian thực (real-time), khả năng xử lý phương tiện (media processing) và độ tin cậy, hệ thống được thiết kế theo \textbf{Kiến trúc 3 Tầng (3-Tier Architecture)} kết hợp với mô hình \textbf{Microservices} ở mức độ Containerization. Kiến trúc này tách biệt rõ ràng giữa giao diện người dùng, logic nghiệp vụ và lưu trữ dữ liệu.

Hệ thống bao gồm các thành phần chính tương tác như sau:
\begin{itemize}
    \item \textbf{Client Layer (SPA):} Ứng dụng ReactJS chạy trên trình duyệt, kết nối đến Server qua REST API (cho các tác vụ stateless) và WebSocket (cho các tác vụ stateful/real-time).
    \item \textbf{Application Server Layer:} Node.js Server xử lý logic chính, bao gồm xác thực, điều phối tin nhắn và xử lý tín hiệu gọi điện (Signaling).
    \item \textbf{Background Worker Layer:} Các tiến trình nền (có thể chạy cùng Node.js hoặc tách riêng) phụ trách việc xử lý video nặng bằng FFmpeg để không làm chặn (blocking) luồng chính.
    \item \textbf{Data Layer:} Bao gồm MongoDB (Primary DB), Redis (Cache/PubSub) và AWS S3 (Object Storage).
\end{itemize}

\subsection{Kiến trúc chi tiết phía Backend}
Backend được tổ chức theo mô hình \textbf{Layered Architecture} chặt chẽ:
\begin{enumerate}
    \item \textbf{Router/Controller:} Tiếp nhận request, validate dữ liệu đầu vào bằng Zod, sau đó chuyển xuống Service.
    \item \textbf{Service Layer:} Chứa toàn bộ business rules. Tại đây, các quy trình phức tạp như "Kiểm tra quyền bạn bè -> Tạo log tin nhắn -> Đẩy thông báo Socket" được thực thi.
    \item \textbf{Data Access Layer (Repository):} Sử dụng Prisma ORM để tương tác với DB. Tầng này giúp ẩn giấu các câu lệnh query phức tạp.
\end{enumerate}

Đặc biệt, module \textbf{Streaming Service} hoạt động độc lập: video sau khi upload sẽ được đưa vào hàng đợi (Queue), sau đó được transcode sang chuẩn HLS (m3u8) và upload lên S3.

\subsection{Biểu đồ thành phần (Component Diagram)}
Hệ thống được triển khai dưới dạng các Docker Container giao tiếp qua Docker Network:
\begin{itemize}
    \item \textbf{Nginx (Gateway):} Reverse Proxy xử lý SSL, Static Files và Load Balancing.
    \item \textbf{App Cluster:} Các instance của Node.js Server (có thể scale horizontal).
    \item \textbf{Redis Adapter:} Đồng bộ hóa sự kiện Socket.io giữa các App Instances.
    \item \textbf{Turn Server (Coturn):} Hỗ trợ relay media WebRTC khi kết nối P2P bị chặn bởi NAT/Firewall.
\end{itemize}

\section{Thiết kế chi tiết}
\subsection{Thiết kế giao diện}
Phần này có độ dài từ hai đến ba trang. Sinh viên đặc tả thông tin về màn hình mà ứng dụng của mình hướng tới, bao gồm độ phân giải màn hình, kích thước màn hình, số lượng màu sắc hỗ trợ, v.v. Tiếp đến, sinh viên đưa ra các thống nhất/chuẩn hóa của mình khi thiết kế giao diện như thiết kế nút, điều khiển, vị trí hiển thị thông điệp phản hồi, phối màu, v.v. Sau cùng sinh viên đưa ra một số hình ảnh minh họa thiết kế giao diện cho các chức năng quan trọng nhất. Lưu ý, sinh viên không nhầm lẫn giao diện thiết kế với giao diện của sản phẩm sau cùng.
\subsection{Thiết kế lớp}
Phần này có độ dài từ ba đến bốn trang. Sinh viên trình bày thiết kế chi tiết các thuộc tính và phương thức cho một số lớp chủ đạo/quan trọng nhất của ứng dụng (từ 2-4 lớp). Thiết kế chi tiết cho các lớp khác, nếu muốn trình bày, sinh viên đưa vào phần phụ lục.

Để minh họa thiết kế lớp, sinh viên thiết kế luồng truyền thông điệp giữa các đối tượng tham gia cho 2 đến 3 use case quan trọng nào đó bằng biểu đồ trình tự (hoặc biểu đồ giao tiếp).
\subsection{Thiết kế cơ sở dữ liệu}
Phần này có độ dài từ hai đến bốn trang. Sinh viên thiết kế, vẽ và giải thích biểu đồ thực thể liên kết (E-R diagram). Từ đó, sinh viên thiết kế cơ sở dữ liệu tùy theo hệ quản trị cơ sở dữ liệu mà mình sử dụng (SQL, NoSQL, Firebase, v.v.)

\section{Xây dựng ứng dụng}

\subsection{Môi trường phát triển và Thư viện}
Hệ thống được phát triển trên môi trường Windows/Linux với các công cụ quản lý mã nguồn Git. Chi tiết các thư viện chính:

\begin{table}[H]
\centering
\caption{Các công nghệ và thư viện cốt lõi}
\begin{tabular}{|l|l|p{6cm}|}
\hline
\textbf{Khu vực} & \textbf{Công nghệ} & \textbf{Vai trò chính} \\ \hline
\multirow{4}{*}{Frontend} & ReactJS 19, Vite & Framework UI và Build tool tối ưu hiệu năng. \\ \cline{2-3}
 & Zustand, React Query & Quản lý Client State và Server State (Caching). \\ \cline{2-3}
 & Socket.io-client & Giao thức thời gian thực, quản lý Room. \\ \cline{2-3}
 & Video.js / HLS.js & Player hỗ trợ phát video streaming HLS. \\ \hline
\multirow{4}{*}{Backend} & Node.js, Express & Nền tảng server xử lý bất đồng bộ. \\ \cline{2-3}
 & Prisma, MongoDB & ORM và Database lưu trữ document. \\ \cline{2-3}
 & WebRTC (SimplePeer) & Thư viện hỗ trợ kết nối P2P cho call video. \\ \cline{2-3}
 & Fluent-ffmpeg & Wrapper cho FFmpeg để xử lý video command. \\ \hline
\multirow{2}{*}{Infra} & Docker Compose & Orchestration cho môi trường dev/prod. \\ \cline{2-3}
 & AWS SDK (S3, SES) & Tích hợp dịch vụ cloud. \\ \hline
\end{tabular}
\end{table}

\subsection{Cài đặt các quy trình nghiệp vụ chính}

\subsubsection{Quy trình xác thực (Authentication)}
Hệ thống sử dụng cơ chế **JWT (JSON Web Token)** kép gồm Access Token (ngắn hạn, 15 phút) và Refresh Token (dài hạn, 7 ngày):
\begin{itemize}
    \item Khi đăng nhập thành công, Server trả về cặp token.
    \item Access Token được dùng để xác thực API Requests.
    \item Khi Access Token hết hạn, Client tự động dùng Refresh Token gọi API `/refresh` để lấy token mới mà không cần user đăng nhập lại (Silent Refresh), đảm bảo trải nghiệm liền mạch.
\end{itemize}

\subsubsection{Quy trình xử lý Video HLS (HTTP Live Streaming)}
Đây là tính năng kỹ thuật phức tạp nhất, nâng cao trải nghiệm xem video:
\begin{enumerate}
    \item **Upload:** Client upload file video gốc (mp4, mov...) lên Server.
    \item **Validation:** Server kiểm tra định dạng, kích thước và virus.
    \item **Transcoding:** FFmpeg được kích hoạt, thực hiện 2 việc:
        \begin{itemize}
            \item Convert video sang codec H.264/AAC.
            \item Cắt video thành các file segments (.ts) độ dài 10s.
            \item Tạo file manifest (.m3u8) chứa thông tin các segments.
        \end{itemize}
    \item **Storage:** Toàn bộ file segment và manifest được upload lên AWS S3.
    \item **Delivery:** Client nhận link `.m3u8`, trình duyệt tự động tải từng segment nhỏ để phát, cho phép tua (seek) nhanh và không cần tải toàn bộ video.
\end{enumerate}

\subsubsection{Thiết lập cuộc gọi WebRTC}
Quy trình Signaling (tín hiệu) để thiết lập kết nối P2P diễn ra qua Socket.io:
\begin{enumerate}
    \item **Offer:** Người gọi (Caller) tạo SDP Offer và gửi qua Socket server.
    \item **Answer:** Người nhận (Callee) nhận Offer, tạo SDP Answer và gửi lại.
    \item **ICE Candidates:** Cả hai bên liên tục trao đổi các ứng viên mạng (IP:Port) để tìm đường đi tốt nhất (qua LAN, Wifi hoặc TURN server).
    \item **Connected:** Khi đường truyền P2P thiết lập thành công, stream Video/Audio được truyền trực tiếp giữa 2 máy.
\end{enumerate}

\subsection{Kết quả đạt được và Minh họa}
Các module đã hoàn thiện và hoạt động ổn định:
\begin{itemize}
    \item \textbf{Module Chat:} Hỗ trợ text, emoji, gửi ảnh, video HLS, file đính kèm. Có hiển thị "typing...", trạng thái Online/Offline.
    \item \textbf{Module Call:} Video Call HD, tự động chuyển đổi camera/micro, giao diện kéo thả (Draggable).
    \item \textbf{Module Friend:} Tìm kiếm người dùng, gửi lời mời kết bạn, chấp nhận/từ chối.
\end{itemize}

\begin{figure}[H]
    \centering
    %\includegraphics[width=0.9\textwidth]{Hinhve/hls_demo.png}
    \caption{Trình phát video HLS trong khung chat, tự động buffer từng segment}
\end{figure}

\section{Kiểm thử và Đánh giá}

\subsection{Phương pháp kiểm thử}
Nhóm thực hiện kiểm thử theo cấp độ **Integration Testing** (Kiểm thử tích hợp) và **System Testing** (Kiểm thử hệ thống).
\begin{itemize}
    \item \textbf{API Testing:} Sử dụng Postman để kiểm tra 100\% các endpoints, đảm bảo Status Code và Response Body chuẩn xác cho các trường hợp Success/Error.
    \item \textbf{Socket Testing:} Giả lập nhiều Client kết nối cùng lúc để test độ trễ của tin nhắn và tính đồng bộ của trạng thái ("Đã xem").
\end{itemize}

\subsection{Kết quả thực nghiệm}
\begin{itemize}
    \item **Độ trễ tin nhắn:** < 100ms trong điều kiện mạng 4G tiêu chuẩn.
    \item **Xử lý Video:** Video 100MB được transcode và sẵn sàng phát sau khoảng 15-20 giây.
    \item **Hiệu năng WebRTC:** Duy trì FPS 24-30 ổn định. Khi Packet Loss > 5\%, chất lượng video tự động giảm bitrate để ưu tiên âm thanh (Audio-first).
\end{itemize}

\section{Triển khai (Deployment)}

\subsection{Mô hình triển khai Docker Compose}
Quy trình triển khai được tự động hóa hoàn toàn bằng file `docker-compose.prod.yml`. Một câu lệnh `docker-compose up -d` sẽ khởi tạo toàn bộ hạ tầng:

\begin{enumerate}
    \item **Service Database:** MongoDB khởi tạo với Replica Set (mô phỏng) để hỗ trợ Transaction.
    \item **Service Redis:** Redis khởi động và load cấu hình persistence.
    \item **Service Backend:** Node.js app chờ DB/Redis sẵn sàng (healthcheck) rồi mới start. Biến môi trường được nạp từ file `.env` bảo mật.
    \item **Service Proxy:** Nginx container mount volume chứng chỉ SSL (Let's Encrypt), cấu hình chặn các request không hợp lệ và cache file tĩnh.
\end{enumerate}

\subsection{Cấu hình Nginx và Bảo mật}
Nginx đóng vai trò "người gác cổng" quan trọng:
\begin{itemize}
    \item **SSL Termination:** Mã hóa toàn bộ traffic bằng HTTPS.
    \item **Client Max Body Size:** Cấu hình giới hạn upload (ví dụ 50MB) để chống tấn công DoS qua upload file.
    \item **Header Security:** Ẩn thông tin Server (X-Powered-By), cấu hình CORS chặt chẽ chỉ cho phép domain của Client truy cập API.
\end{itemize}

\end{document}
