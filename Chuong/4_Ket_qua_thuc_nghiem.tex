\documentclass[../DoAn.tex]{subfiles}
\begin{document}

\section{Thiết kế kiến trúc hệ thống}

\subsection{Mô hình tổng quan}
Dựa trên yêu cầu về tính thời gian thực (real-time), khả năng xử lý phương tiện (media processing) và độ tin cậy, hệ thống được thiết kế theo \textbf{Kiến trúc 3 Tầng (3-Tier Architecture)} kết hợp với mô hình \textbf{Microservices} ở mức độ Containerization. Kiến trúc này tách biệt rõ ràng giữa giao diện người dùng, logic nghiệp vụ và lưu trữ dữ liệu.

Hệ thống bao gồm các thành phần chính tương tác như sau:
\begin{itemize}
    \item \textbf{Client Layer (SPA):} Ứng dụng ReactJS chạy trên trình duyệt, kết nối đến Server qua REST API (cho các tác vụ stateless) và WebSocket (cho các tác vụ stateful/real-time).
    \item \textbf{Application Server Layer:} Node.js Server xử lý logic chính, bao gồm xác thực, điều phối tin nhắn và xử lý tín hiệu gọi điện (Signaling).
    \item \textbf{Background Worker Layer:} Các tiến trình nền (có thể chạy cùng Node.js hoặc tách riêng) phụ trách việc xử lý video nặng bằng FFmpeg để không làm chặn (blocking) luồng chính.
    \item \textbf{Data Layer:} Bao gồm MongoDB (Primary DB), Redis (Cache/PubSub) và AWS S3 (Object Storage).
\end{itemize}

\begin{figure}[H]
    \centering
    \includegraphics[width=1.0\textwidth]{Hinhve/System_Overview_Architecture.png}
    \caption{Mô hình kiến trúc tổng quan 3 tầng (3-Tier Architecture)}
    \label{fig:system_overview}
\end{figure}

\subsection{Kiến trúc chi tiết phía Backend}
Backend được tổ chức theo mô hình \textbf{Layered Architecture} chặt chẽ:
\begin{enumerate}
    \item \textbf{Router/Controller:} Tiếp nhận request, validate dữ liệu đầu vào bằng Zod, sau đó chuyển xuống Service.
    \item \textbf{Service Layer:} Chứa toàn bộ business rules. Tại đây, các quy trình phức tạp như "Kiểm tra quyền bạn bè -> Tạo log tin nhắn -> Đẩy thông báo Socket" được thực thi.
    \item \textbf{Data Access Layer (Repository):} Sử dụng Prisma ORM để tương tác với DB. Tầng này giúp ẩn giấu các câu lệnh query phức tạp.
\end{enumerate}

Đặc biệt, module \textbf{Streaming Service} hoạt động độc lập: video sau khi upload sẽ được đưa vào hàng đợi (Queue), sau đó được transcode sang chuẩn HLS (m3u8) và upload lên S3.

\begin{figure}[H]
    \centering
    \includegraphics[width=1.0\textwidth]{Hinhve/Backend_Layered_Architecture.png} 
    \caption{Kiến trúc chi tiết Backend theo mô hình Layered}
    \label{fig:backend_layered}
\end{figure}


\section{Thiết kế chi tiết}
\subsection{Thiết kế giao diện}
Phần này có độ dài từ hai đến ba trang. Sinh viên đặc tả thông tin về màn hình mà ứng dụng của mình hướng tới, bao gồm độ phân giải màn hình, kích thước màn hình, số lượng màu sắc hỗ trợ, v.v. Tiếp đến, sinh viên đưa ra các thống nhất/chuẩn hóa của mình khi thiết kế giao diện như thiết kế nút, điều khiển, vị trí hiển thị thông điệp phản hồi, phối màu, v.v. Sau cùng sinh viên đưa ra một số hình ảnh minh họa thiết kế giao diện cho các chức năng quan trọng nhất. Lưu ý, sinh viên không nhầm lẫn giao diện thiết kế với giao diện của sản phẩm sau cùng.
\subsection{Thiết kế lớp (Class Design)}
Dựa trên kiến trúc Layered Architecture và Microservices, hệ thống được chia thành nhiều lớp xử lý chuyên biệt. Dưới đây là thiết kế chi tiết các lớp Service quan trọng nhất, nơi chứa phần lớn logic nghiệp vụ (Business Logic) của hệ thống.
\begin{figure}[H]
    \centering
    \includegraphics[width=1.0\textwidth]{Hinhve/class_diagram_core.png}
    \caption{Thiết kế lớp (Class Design)}
    \label{fig:class_diagram_core}
\end{figure}
\subsubsection{Chi tiết các lớp xử lý chính}

\paragraph{1. AuthService (Dịch vụ xác thực)}
Lớp \textbf{AuthService} chịu trách nhiệm quản lý phiên làm việc của người dùng, bao gồm đăng ký, đăng nhập và cơ chế cấp phát lại Token (Refresh Token). Đây là lớp bảo mật cốt lõi, đảm bảo rằng chỉ những người dùng hợp lệ mới có thể truy cập hệ thống.

\begin{itemize}
    \item \textbf{Các thuộc tính (Dependencies):}
    \begin{itemize}
        \item \texttt{userService}: Tương tác với cơ sở dữ liệu User.
        \item \texttt{jwtService}: Mã hóa và giải mã Access/Refresh Token.
        \item \texttt{redisService}: Lưu trữ Whitelist/Blacklist token để quản lý phiên đăng nhập.
    \end{itemize}
    \item \textbf{Các phương thức chính:}
    \begin{itemize}
        \item \texttt{register(registerToken, fields, passwordRaw)}: Hoàn tất đăng ký người dùng sau khi xác thực OTP thành công.
        \item \texttt{login(email, passwordRaw)}: Kiểm tra thông tin đăng nhập và trả về cặp Access/Refresh Token.
        \item \texttt{refreshToken(token)}: Kiểm tra tính hợp lệ của Refresh Token trong Redis và cấp phát Access Token mới (Cơ chế Silent Refresh).
        \item \texttt{logout(refreshToken)}: Hủy phiên làm việc hiện tại bằng cách xóa token khỏi Redis.
        \item \texttt{verifyEmail(email, otp)}: Xác minh địa chỉ email người dùng.
    \end{itemize}
\end{itemize}

\paragraph{2. MediaService (Dịch vụ đa phương tiện)}
Lớp \textbf{MediaService} xử lý các tác vụ liên quan đến upload và chuyển đổi định dạng file. Đặc biệt, lớp này tích hợp quy trình transcode video sang chuẩn HLS để phục vụ streaming.

\begin{itemize}
    \item \textbf{Các thuộc tính (Dependencies):}
    \begin{itemize}
        \item \texttt{prismaService}: Lưu trữ metadata của file vào database.
        \item \texttt{mediaQueue}: Hàng đợi (Queue) để xử lý các tác vụ nặng (transcode video) ở background worker.
        \item \texttt{s3Service}: Interface giao tiếp với AWS S3 Object Storage.
    \end{itemize}
    \item \textbf{Các phương thức chính:}
    \begin{itemize}
        \item \texttt{handleVideoToHLS(videoFile)}: Tiếp nhận video gốc, tạo bản ghi trạng thái "Processing" và đẩy vào hàng đợi xử lý.
        \item \texttt{serveVideoM3u8(id)}: Trả về file manifest (.m3u8) để client bắt đầu luồng phát video.
        \item \texttt{multiUploadAndCreateMessage(files, ...)}: Xử lý upload đồng thời nhiều ảnh/file và tự động tạo tin nhắn tương ứng.
        \item \texttt{handleTransformFile(files)}: Chuẩn hóa file (ví dụ: resize ảnh, convert audio) trước khi lưu trữ để tối ưu dung lượng.
    \end{itemize}
\end{itemize}

\paragraph{3. SocketService (Dịch vụ thời gian thực)}
Lớp \textbf{SocketService} đóng vai trò là tầng giao tiếp (Communication Layer), quản lý các kết nối WebSocket và phát tán sự kiện (Events) tới Client.

\begin{itemize}
    \item \textbf{Các thuộc tính (Dependencies):}
    \begin{itemize}
        \item \texttt{io}: Instance của Socket.io Server.
    \end{itemize}
    \item \textbf{Các phương thức chính:}
    \begin{itemize}
        \item \texttt{joinChat(chatId, userIds)}: Đưa người dùng vào các "Room" chat riêng biệt để nhận tin nhắn.
        \item \texttt{sendMessage(chatId, payload)}: Gửi sự kiện tin nhắn mới tới tất cả thành viên trong Room.
        \item \texttt{mediaProcessingUpdate(chatId, media)}: Thông báo realtime cho Client khi video đã xử lý xong (từ trang thái Processing -> Ready).
        \item \texttt{memberAdded/Removed(...)}: Cập nhật danh sách thành viên trong nhóm theo thời gian thực.
    \end{itemize}
\end{itemize}

\subsubsection{Biểu đồ tương tác (Sequence Diagrams)}

\paragraph{1. Quy trình Gửi tin nhắn video và Xử lý HLS}
Biểu đồ dưới đây mô tả luồng dữ liệu khi người dùng gửi một video. Hệ thống tách biệt việc upload và việc xử lý video (Transcoding) để đảm bảo phản hồi nhanh cho người dùng.

\begin{figure}[H]
    \centering
    \includegraphics[width=0.95\textwidth]{Hinhve/Seq_SendVideoHLS.png}
    \caption{Biểu đồ tuần tự use case Gửi Video và Xử lý HLS Background}
    \label{fig:seq_video}
\end{figure}

\paragraph{2. Quy trình Đăng nhập và Tự động cấp lại Token}
Mô tả cơ chế bảo mật sử dụng JWT và Redis. Khi Access Token hết hạn, Client tự động yêu cầu cấp mới mà không làm gián đoạn trải nghiệm người dùng.

\begin{figure}[H]
    \centering
    \includegraphics[width=0.95\textwidth]{Hinhve/Seq_LoginRefresh.png}
    \caption{Biểu đồ tuần tự use case Đăng nhập và Refresh Token}
    \label{fig:seq_auth}
\end{figure}
\subsection{Thiết kế cơ sở dữ liệu}
Hệ thống sử dụng hệ quản trị cơ sở dữ liệu \textbf{MongoDB} (NoSQL). Việc lựa chọn NoSQL thay vì RDBMS truyền thống (như MySQL, PostgreSQL) dựa trên các yếu tố sau:
\begin{itemize}
    \item \textbf{High Write Throughput:} Ứng dụng chat real-time tạo ra lượng lớn tin nhắn trong thời gian ngắn, MongoDB tối ưu tốt cho việc ghi dữ liệu liên tục.
    \item \textbf{Flexible Schema:} Cấu trúc tin nhắn có thể thay đổi linh hoạt (text, hình ảnh, video, notification) mà không cần migrate schema phức tạp.
    \item \textbf{Denormalization:} Khả năng nhúng (embedding) dữ liệu giúp giảm thiểu các truy vấn JOIN đắt đỏ, tăng tốc độ đọc cho các tính năng như "Danh sách tin nhắn gần nhất".
\end{itemize}

\subsubsection{Biểu đồ Quan hệ (Entity-Relationship Diagram)}
Mặc dù là NoSQL, hệ thống vẫn duy trì các mối quan hệ chặt chẽ giữa các thực thể thông qua `ObjectId`.

\begin{figure}[H]
    \centering
    \includegraphics[width=1.0\textwidth]{Hinhve/Database_ERD.png} 
    \caption{Biểu đồ quan hệ thực thể (ERD) của hệ thống}
    \label{fig:db_erd}
\end{figure}

\subsubsection{Chi tiết các Collections chính}

Dưới đây là đặc tả chi tiết các Collection quan trọng trong cơ sở dữ liệu:

\paragraph{1. Collection Users}
Lưu trữ thông tin định danh và hồ sơ người dùng. Các trường `email` và `username` được đánh Index Unique để đảm bảo tính duy nhất.

\begin{table}[H]
    \centering
    \caption{Cấu trúc Collection Users}
    \begin{tabular}{|l|l|p{8cm}|}
    \hline
    \textbf{Trường} & \textbf{Kiểu} & \textbf{Mô tả} \\ \hline
    \_id & ObjectId & Khóa chính (Primary Key). \\ \hline
    email & String & Email đăng nhập (Unique). \\ \hline
    username & String & Tên định danh (Unique, Fulltext Index). \\ \hline
    hashedPassword & String & Mật khẩu đã mã hóa (Bcrypt). \\ \hline
    displayName & String & Tên hiển thị với người dùng khác. \\ \hline
    avatarUrl & String & Đường dẫn ảnh đại diện. \\ \hline
    createdAt & Date & Thời gian tạo tài khoản. \\ \hline
    \end{tabular}
\end{table}

\paragraph{2. Collection Chats}
Lưu trữ thông tin về các cuộc hội thoại (Direct hoặc Group). Điểm đặc biệt là trường `lastMessage` và `groupInfo` được thiết kế dạng \textbf{Embedded Document} để tối ưu hiệu năng khi hiển thị danh sách chat.

\begin{table}[H]
    \centering
    \caption{Cấu trúc Collection Chats}
    \begin{tabular}{|l|l|p{8cm}|}
    \hline
    \textbf{Trường} & \textbf{Kiểu} & \textbf{Mô tả} \\ \hline
    \_id & ObjectId & Khóa chính. \\ \hline
    type & Enum & Loại chat (`direct` hoặc `group`). \\ \hline
    lastMessage & Object & \textit{(Embedded)} Chứa thông tin tóm tắt tin nhắn cuối cùng (content, senderId, createdAt). Giúp client hiển thị preview mà không cần query vào bảng Message. \\ \hline
    groupInfo & Object & \textit{(Embedded)} Chứa tên nhóm, người tạo (chỉ có khi type=group). \\ \hline
    participantIds & Array & Danh sách ID người tham gia (Relation). \\ \hline
    \end{tabular}
\end{table}

\paragraph{3. Collection Messages}
Lưu trữ nội dung chi tiết của tin nhắn. Mỗi tin nhắn thuộc về một Chat và một Sender cụ thể.

\begin{table}[H]
    \centering
    \caption{Cấu trúc Collection Messages}
    \begin{tabular}{|l|l|p{8cm}|}
    \hline
    \textbf{Trường} & \textbf{Kiểu} & \textbf{Mô tả} \\ \hline
    \_id & ObjectId & Khóa chính. \\ \hline
    chatId & ObjectId & Reference tới Collection Chats. \\ \hline
    senderId & ObjectId & Reference tới Collection Users. \\ \hline
    content & String & Nội dung văn bản của tin nhắn. \\ \hline
    medias & Array & Danh sách các file đính kèm (Media Objects). \\ \hline
    createdAt & Date & Thời gian gửi (Index Time-to-Live nếu cần). \\ \hline
    \end{tabular}
\end{table}

\paragraph{4. Collection Medias}
Quản lý các tệp đa phương tiện. Với video, hệ thống lưu trạng thái xử lý để phục vụ quy trình transcode HLS.

\begin{table}[H]
    \centering
    \caption{Cấu trúc Collection Medias}
    \begin{tabular}{|l|l|p{8cm}|}
    \hline
    \textbf{Trường} & \textbf{Kiểu} & \textbf{Mô tả} \\ \hline
    \_id & ObjectId & Khóa chính. \\ \hline
    type & Enum & Loại file (`image`, `video`, `video\_hls`, `file`). \\ \hline
    url & String & Đường dẫn tới file trên AWS S3. \\ \hline
    status & Enum & Trạng thái xử lý (`pending`, `processing`, `completed`, `failed`). \\ \hline
    messageId & ObjectId & Reference tới tin nhắn chứa media này. \\ \hline
    \end{tabular}
\end{table}

\section{Xây dựng ứng dụng}

\subsection{Môi trường phát triển và Thư viện}
Hệ thống được phát triển trên môi trường Windows/Linux với các công cụ quản lý mã nguồn Git. Chi tiết các thư viện chính:

\begin{table}[H]
\centering
\caption{Các công nghệ và thư viện cốt lõi}
\begin{tabular}{|l|l|p{6cm}|}
\hline
\textbf{Khu vực} & \textbf{Công nghệ} & \textbf{Vai trò chính} \\ \hline
\multirow{4}{*}{Frontend} & ReactJS 19, Vite & Framework UI và Build tool tối ưu hiệu năng. \\ \cline{2-3}
 & Zustand, React Query & Quản lý Client State và Server State (Caching). \\ \cline{2-3}
 & Socket.io-client & Giao thức thời gian thực, quản lý Room. \\ \cline{2-3}
 & Video.js / HLS.js & Player hỗ trợ phát video streaming HLS. \\ \hline
\multirow{4}{*}{Backend} & Node.js, Express & Nền tảng server xử lý bất đồng bộ. \\ \cline{2-3}
 & Prisma, MongoDB & ORM và Database lưu trữ document. \\ \cline{2-3}
 & WebRTC (SimplePeer) & Thư viện hỗ trợ kết nối P2P cho call video. \\ \cline{2-3}
 & Fluent-ffmpeg & Wrapper cho FFmpeg để xử lý video command. \\ \hline
\multirow{2}{*}{Infra} & Docker Compose & Orchestration cho môi trường dev/prod. \\ \cline{2-3}
 & AWS SDK (S3, SES) & Tích hợp dịch vụ cloud. \\ \hline
\end{tabular}
\end{table}

\subsection{Cài đặt các quy trình nghiệp vụ chính}

\subsubsection{Quy trình xác thực (Authentication)}
Hệ thống sử dụng cơ chế **JWT (JSON Web Token)** kép gồm Access Token (ngắn hạn, 15 phút) và Refresh Token (dài hạn, 7 ngày):
\begin{itemize}
    \item Khi đăng nhập thành công, Server trả về cặp token.
    \item Access Token được dùng để xác thực API Requests.
    \item Khi Access Token hết hạn, Client tự động dùng Refresh Token gọi API `/refresh` để lấy token mới mà không cần user đăng nhập lại (Silent Refresh), đảm bảo trải nghiệm liền mạch.
    
    \begin{figure}[H]
        \centering
        \includegraphics[width=0.9\textwidth]{Hinhve/Seq_JWT_Auth_Refresh.png}
        \caption{Quy trình xác thực JWT và cơ chế Refresh Token tự động}
    \end{figure}
\end{itemize}

\subsubsection{Quy trình xử lý Video HLS (HTTP Live Streaming)}
Đây là tính năng kỹ thuật phức tạp nhất, nâng cao trải nghiệm xem video:
\begin{enumerate}
    \item **Upload:** Client upload file video gốc (mp4, mov...) lên Server.
    \item **Validation:** Server kiểm tra định dạng, kích thước và virus.
    \item **Transcoding:** FFmpeg được kích hoạt, thực hiện 2 việc:
        \begin{itemize}
            \item Convert video sang codec H.264/AAC.
            \item Cắt video thành các file segments (.ts) độ dài 10s.
            \item Tạo file manifest (.m3u8) chứa thông tin các segments.
        \end{itemize}
    \item **Storage:** Toàn bộ file segment và manifest được upload lên AWS S3.
    \item **Delivery:** Client nhận link `.m3u8`, trình duyệt tự động tải từng segment nhỏ để phát, cho phép tua (seek) nhanh và không cần tải toàn bộ video.
    
    \begin{figure}[H]
        \centering
        \includegraphics[width=0.9\textwidth]{Hinhve/Seq_HLS_Workflow.png}
        \caption{Quy trình xử lý Video HLS (HTTP Live Streaming)}
    \end{figure}
\end{enumerate}

\subsubsection{Thiết lập cuộc gọi WebRTC}
Quy trình Signaling (tín hiệu) để thiết lập kết nối P2P diễn ra qua Socket.io:
\begin{enumerate}
    \item **Offer:** Người gọi (Caller) tạo SDP Offer và gửi qua Socket server.
    \item **Answer:** Người nhận (Callee) nhận Offer, tạo SDP Answer và gửi lại.
    \item **ICE Candidates:** Cả hai bên liên tục trao đổi các ứng viên mạng (IP:Port) để tìm đường đi tốt nhất (qua LAN, Wifi hoặc TURN server).
    \item **Connected:** Khi đường truyền P2P thiết lập thành công, stream Video/Audio được truyền trực tiếp giữa 2 máy.
    
    \begin{figure}[H]
        \centering
        \includegraphics[width=0.9\textwidth]{Hinhve/Seq_WebRTC_Setup.png}
        \caption{Quy trình thiết lập cuộc gọi WebRTC}
    \end{figure}
\end{enumerate}

\subsection{Kết quả đạt được và Minh họa}
Các module đã hoàn thiện và hoạt động ổn định:
\begin{itemize}
    \item \textbf{Module Chat:} Hỗ trợ text, emoji, gửi ảnh, video HLS, file đính kèm. Có hiển thị "typing...", trạng thái Online/Offline.
    \item \textbf{Module Call:} Video Call HD, tự động chuyển đổi camera/micro, giao diện kéo thả (Draggable).
    \item \textbf{Module Friend:} Tìm kiếm người dùng, gửi lời mời kết bạn, chấp nhận/từ chối.
\end{itemize}

\begin{figure}[H]
    \centering
    \includegraphics[width=0.9\textwidth]{Hinhve/hls_demo.png}
    \caption{Trình phát video HLS trong khung chat, tự động buffer từng segment}
\end{figure}

\section{Kiểm thử và Đánh giá}

\subsection{Phương pháp kiểm thử}
Nhóm thực hiện kiểm thử theo cấp độ **Integration Testing** (Kiểm thử tích hợp) và **System Testing** (Kiểm thử hệ thống).
\begin{itemize}
    \item \textbf{API Testing:} Sử dụng Postman để kiểm tra 100\% các endpoints, đảm bảo Status Code và Response Body chuẩn xác cho các trường hợp Success/Error.
    \item \textbf{Socket Testing:} Giả lập nhiều Client kết nối cùng lúc để test độ trễ của tin nhắn và tính đồng bộ của trạng thái ("Đã xem").
\end{itemize}

\subsection{Kết quả thực nghiệm}
\begin{itemize}
    \item **Độ trễ tin nhắn:** < 100ms trong điều kiện mạng 4G tiêu chuẩn.
    \item **Xử lý Video:** Video 100MB được transcode và sẵn sàng phát sau khoảng 15-20 giây.
    \item **Hiệu năng WebRTC:** Duy trì FPS 24-30 ổn định. Khi Packet Loss > 5\%, chất lượng video tự động giảm bitrate để ưu tiên âm thanh (Audio-first).
\end{itemize}

\section{Triển khai (Deployment)}

\subsection{Mô hình triển khai Docker Compose}
Quy trình triển khai được tự động hóa hoàn toàn bằng file `docker-compose.prod.yml`. Một câu lệnh `docker-compose up -d` sẽ khởi tạo toàn bộ hạ tầng:

\begin{enumerate}
    \item **Service Database:** MongoDB khởi tạo với Replica Set (mô phỏng) để hỗ trợ Transaction.
    \item **Service Redis:** Redis khởi động và load cấu hình persistence.
    \item **Service Backend:** Node.js app chờ DB/Redis sẵn sàng (healthcheck) rồi mới start. Biến môi trường được nạp từ file `.env` bảo mật.
    \item **Service Proxy:** Nginx container mount volume chứng chỉ SSL (Let's Encrypt), cấu hình chặn các request không hợp lệ và cache file tĩnh.
\end{enumerate}

\subsection{Cấu hình Nginx và Bảo mật}
Nginx đóng vai trò "người gác cổng" quan trọng:
\begin{itemize}
    \item **SSL Termination:** Mã hóa toàn bộ traffic bằng HTTPS.
    \item **Client Max Body Size:** Cấu hình giới hạn upload (ví dụ 50MB) để chống tấn công DoS qua upload file.
    \item **Header Security:** Ẩn thông tin Server (X-Powered-By), cấu hình CORS chặt chẽ chỉ cho phép domain của Client truy cập API.
\end{itemize}

\end{document}
