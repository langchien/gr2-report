\documentclass[../DoAn.tex]{subfiles}
\begin{document}
Chương này sẽ tập trung vào việc khảo sát các hệ thống tương tự, phân tích các yêu cầu chức năng và phi chức năng của hệ thống Chat App. Từ việc khảo sát, đồ án sẽ đi sâu vào thiết kế tổng quan các chức năng thông qua biểu đồ Use Case, đồng thời đặc tả chi tiết các quy trình nghiệp vụ chính để làm cơ sở cho việc thiết kế và cài đặt ở các chương sau.

\section{Khảo sát hiện trạng}
\label{section:2.1}
Hiện nay, nhu cầu trao đổi thông tin trực tuyến đóng vai trò thiết yếu trong đời sống và công việc. Các ứng dụng nhắn tin tức thời (Instant Messaging) không chỉ đơn thuần là gửi văn bản mà còn tích hợp đa phương tiện, gọi thoại/video chất lượng cao và khả năng làm việc nhóm.

Qua khảo sát các ứng dụng phổ biến trên thị trường như Facebook Messenger, Zalo, và Telegram, nhóm nhận thấy các ứng dụng này đều có điểm chung là tập trung vào trải nghiệm người dùng mượt mà, tốc độ phản hồi nhanh (real-time) và khả năng xử lý dữ liệu đa phương tiện tốt. Tuy nhiên, việc xây dựng một hệ thống tương tự đòi hỏi sự kết hợp phức tạp của nhiều công nghệ hiện đại.

Dựa trên khảo sát và mục tiêu nghiên cứu, đồ án tập trung xây dựng một ứng dụng Chat trên nền web với các nhóm tính năng quan trọng sau:
\begin{itemize}
    \item \textbf{Xác thực \& Người dùng}: Đảm bảo bảo mật tài khoản, hỗ trợ đăng nhập đa nền tảng (OAuth2 Google), quản lý thông tin cá nhân và trạng thái hoạt động (Online/Offline).
    \item \textbf{Mạng xã hội (Social)}: Xây dựng mạng lưới kết nối thông qua tìm kiếm, kết bạn, quản lý danh sách bạn bè và tạo nhóm chat.
    \item \textbf{Nhắn tin (Messaging)}: Hỗ trợ trao đổi tin nhắn văn bản, biểu cảm (Emoji), và đặc biệt là chia sẻ tài liệu, hình ảnh, video (hỗ trợ Streaming HLS cho video dung lượng lớn).
    \item \textbf{Gọi điện (Calling)}: Tích hợp tính năng gọi Video Call trực tiếp giữa người dùng thông qua công nghệ WebRTC.
    \item \textbf{Hệ thống}: Các tính năng nền tảng như thông báo thời gian thực (Notification) và tìm kiếm toàn cục.
\end{itemize}

\section{Tổng quan chức năng}
\label{section:2.2}
Phần này tóm tắt các nhóm chức năng chính mà hệ thống hướng tới, đáp ứng nhu cầu giao tiếp toàn diện của người dùng. Hệ thống không chỉ cung cấp các tính năng nhắn tin cơ bản mà còn tích hợp các tiện ích nâng cao như gọi video chất lượng cao, chia sẻ tệp tin dung lượng lớn và quản lý nhóm hiệu quả. Các chức năng được chia thành các phân hệ rõ ràng để thuận tiện cho việc thiết kế và phát triển.

\subsection{Biểu đồ use case tổng quát}
\label{subsection:2.2.1}
Hệ thống được thiết kế xoay quanh tác nhân chính là \textbf{Người dùng (User)}. Dưới đây là biểu đồ Use Case tổng quát mô tả các chức năng chính của hệ thống:

\begin{figure}[H]
    \centering
    \includegraphics[width=1.0\textwidth]{Hinhve/Biểu đồ use case tổng quát.png}
    \caption{Biểu đồ Use Case tổng quát của hệ thống}
    \label{fig:usecase_tongquat}
\end{figure}

\textbf{Danh sách các tác nhân:}
\begin{itemize}
    \item \textbf{Người dùng (User)}: Là người đã đăng ký tài khoản thành công và đăng nhập vào hệ thống. Người dùng có toàn quyền sử dụng các chức năng giao tiếp và quản lý cá nhân.
\end{itemize}

\textbf{Mô tả các nhóm Use Case chính:}
\begin{enumerate}
    \item \textbf{Nhóm Xác thực (Authentication)}:
    \begin{itemize}
        \item Đăng ký, Đăng nhập (tài khoản thường và Google OAuth2).
        \item Quên mật khẩu, Đổi mật khẩu.
        \item Cập nhật Profile (Avatar, thông tin cá nhân) và quản lý trạng thái Online/Offline.
    \end{itemize}
    
    \item \textbf{Nhóm Bạn bè \& Tìm kiếm (Social)}:
    \begin{itemize}
        \item Tìm kiếm người dùng khác trong hệ thống.
        \item Gửi lời mời kết bạn, Chấp nhận hoặc Từ chối lời mời.
        \item Xem danh sách bạn bè và Hủy kết bạn.
    \end{itemize}
    
    \item \textbf{Nhóm Nhóm \& Trò chuyện (Group \& Chat)}:
    \begin{itemize}
        \item Tạo cuộc hội thoại mới hoặc mở cuộc hội thoại đã có.
        \item Tạo nhóm chat mới, xem danh sách nhóm.
        \item Quản lý thành viên nhóm (Thêm, Xóa thành viên).
        \item Xóa cuộc hội thoại lịch sử.
    \end{itemize}
    
    \item \textbf{Nhóm Nhắn tin \& Gọi điện (Messaging \& Calling)}:
    \begin{itemize}
        \item Gửi nhận tin nhắn văn bản (Text), biểu tượng cảm xúc (Emoji).
        \item Gửi nhận tệp tin đa phương tiện: Ảnh, Video, Audio.
        \item Xem video chất lượng cao thông qua luồng phát trực tuyến (HLS Streaming).
        \item Thực hiện cuộc gọi Video trực tuyến (Video Call) sử dụng WebRTC.
    \end{itemize}
\end{enumerate}

\subsection{Biểu đồ use case phân rã XYZ}
\label{subsection:2.2.2}
Với mỗi use case mức cao trong biểu đồ use case tổng quan, sinh viên tạo một mục riêng như mục \ref{subsection:2.2.2} và tiến hành phân rã use case đó. Lưu ý tên use case cần phân rã trong biểu đồ use case tổng quan phải khớp với tên đề mục.

Trong mỗi mục như vậy, sinh viên vẽ và giải thích ngắn gọn các use case phân rã.

\subsection{Quy trình nghiệp vụ}
\label{subsection:2.2.3}
Nếu sản phẩm/hệ thống cần xây dựng có quy trình nghiệp vụ quan trọng/đáng chú ý, sinh viên cần mô tả và vẽ biểu đồ hoạt động minh họa quy trình nghiệp vụ đó. Sinh viên lưu ý đây không phải là luồng sự kiện của từng use case, mà là luồng hoạt động kết hợp nhiều use case để thực hiện một nghiệp vụ nào đó.

Ví dụ, một hệ thống quản lý thư viện có quy trình nghiệp vụ mượn trả với mô tả sơ bộ như sau: Sinh viên làm thẻ mượn, sau đó sinh viên đăng ký mượn sách, thủ thư cho mượn, và cuối cùng sinh viên trả lại sách cho thư viện. Một hệ thống có thể có một vài quy trình nghiệp vụ quan trọng như vậy.
\section{Đặc tả chức năng}
\label{section:2.3}
Sinh viên lựa chọn từ 4 đến 7 use case quan trọng nhất của đồ án để đặc tả chi tiết. Mỗi đặc tả bao gồm ít nhất các thông tin sau: (i) Tên use case, (ii) Luồng sự kiện (chính và phát sinh), (iii) Tiền điều kiện, và (iv) Hậu điều kiện. Sinh viên chỉ vẽ bổ sung biểu đồ hoạt động khi đặc tả use case phức tạp.
\subsection{Đặc tả use case A}
\hfill
\subsection{Đặc tả use case B}
\hfill

\section{Yêu cầu phi chức năng}
\label{section:2.4}
Các yêu cầu phi chức năng đóng vai trò quan trọng trong việc đảm bảo chất lượng và trải nghiệm người dùng của hệ thống. Dưới đây là các yêu cầu chi tiết:

\begin{itemize}
    \item \textbf{Hiệu năng (Performance)}:
    \begin{itemize}
        \item Hệ thống phải đảm bảo tính thời gian thực (Real-time). Độ trễ khi gửi/nhận tin nhắn không được vượt quá 200ms trong điều kiện mạng ổn định.
        \item Video streaming phải mượt mà, hỗ trợ tự động điều chỉnh chất lượng (Adaptive Bitrate Streaming) tùy theo băng thông mạng của người dùng.
        \item Thời gian tải trang ban đầu không quá 3 giây.
    \end{itemize}
    
    \item \textbf{Bảo mật (Security)}:
    \begin{itemize}
        \item Tài khoản người dùng phải được bảo vệ, mật khẩu lưu trữ trong cơ sở dữ liệu phải được mã hóa (hashing).
        \item Giao tiếp giữa Client và Server phải được mã hóa thông qua giao thức HTTPS và WSS (Secure WebSocket).
        \item Cơ chế xác thực qua Token (JWT) phải đảm bảo tính toàn vẹn và hết hạn hợp lý.
        \item Dữ liệu riêng tư của người dùng không được rò rỉ hoặc truy cập trái phép.
    \end{itemize}
    
    \item \textbf{Tính khả dụng và Trải nghiệm người dùng (Usability)}:
    \begin{itemize}
        \item Giao diện phải thân thiện, dễ sử dụng, hỗ trợ Responsive để hiển thị tốt trên cả máy tính và thiết bị di động.
        \item Hệ thống phải có cơ chế phản hồi rõ ràng cho các hành động của người dùng (ví dụ: thông báo loading, thông báo lỗi, thông báo thành công).
    \end{itemize}
    
    \item \textbf{Độ tin cậy (Reliability)}:
    \begin{itemize}
        \item Hệ thống hoạt động ổn định 24/7.
        \item Dữ liệu tin nhắn và tệp tin không được thất lạc trong quá trình truyền tải.
        \item Hệ thống có khả năng phục hồi sau khi gặp sự cố gián đoạn kết nối mạng.
    \end{itemize}
    
    \item \textbf{Khả năng bảo trì và mở rộng (Maintainability \& Scalability)}:
    \begin{itemize}
        \item Mã nguồn phải được tổ chức rõ ràng theo kiến trúc định sẵn (ví dụ: Client-Server, MVC) để dễ dàng nâng cấp và sửa lỗi.
        \item Hệ thống có khả năng mở rộng để hỗ trợ thêm tính năng mới mà không làm ảnh hưởng đến các chức năng hiện có.
        \item Sử dụng các công nghệ containerization (Docker) để dễ dàng triển khai trên các môi trường khác nhau.
    \end{itemize}
\end{itemize}


%%%%%%%%%%%%%%%%%%%%%%%%%%%%%%%%%%%

\end{document}