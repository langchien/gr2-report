\documentclass[../DoAn.tex]{subfiles}
\begin{document}

\section{Đặt vấn đề}
\label{section:1.1}
Trong kỷ nguyên số hóa hiện nay, nhu cầu kết nối và trao đổi thông tin trực tuyến đóng vai trò then chốt trong mọi hoạt động kinh tế - xã hội. Sự bùng nổ của các ứng dụng nhắn tin và gọi điện theo thời gian thực (Real-time Communication) đã thay đổi hoàn toàn thói quen giao tiếp của con người, xóa bỏ rào cản về khoảng cách địa lý.

Tuy nhiên, việc phụ thuộc vào các nền tảng thương mại có sẵn đôi khi đặt ra những lo ngại về quyền riêng tư dữ liệu, khả năng tùy biến cho các nhu cầu đặc thù của tổ chức, hoặc đơn giản là giới hạn trong việc tích hợp sâu vào các hệ thống nghiệp vụ nội bộ. Hơn nữa, dưới góc độ kỹ thuật, việc xây dựng một hệ thống truyền tải dữ liệu đa phương tiện (âm thanh, hình ảnh) với độ trễ thấp và tính ổn định cao luôn là một thách thức lớn, đòi hỏi sự am hiểu sâu sắc về các giao thức mạng hiện đại.

Xuất phát từ thực tế đó, đề tài tập trung nghiên cứu và xây dựng ứng dụng web nhắn tin, gọi điện nhằm giải quyết bài toán về làm chủ công nghệ truyền tin thời gian thực, đồng thời cung cấp một giải pháp giao tiếp trực tuyến hiệu quả, linh hoạt và có khả năng mở rộng cao.

\section{Mục tiêu và phạm vi đề tài}
\label{section:1.2}
Mục tiêu chính của đề tài là nghiên cứu và làm chủ các công nghệ cốt lõi trong việc xây dựng hệ thống giao tiếp thời gian thực, từ đó phát triển một ứng dụng web hoàn chỉnh hỗ trợ nhắn tin và gọi điện video. Cụ thể, đề tài hướng tới các mục tiêu sau:
\begin{itemize}
    \item Nghiên cứu cơ chế hoạt động của giao thức WebSocket để thực hiện tính năng nhắn tin tức thời (Instant Messaging).
    \item Tìm hiểu và ứng dụng công nghệ WebRTC (Web Real-Time Communication) để triển khai tính năng gọi thoại và gọi video trực tiếp trên trình duyệt mà không cần cài đặt thêm phần mềm hỗ trợ.
    \item Xây dựng hệ thống backend có khả năng xử lý đồng thời nhiều kết nối, đảm bảo tính ổn định và tốc độ truyền tải nhanh.
\end{itemize}

Về phạm vi, đề tài tập trung vào việc phát triển các chức năng thiết yếu nhất của một ứng dụng giao tiếp hiện đại:
\begin{itemize}
    \item \textbf{Quản lý tài khoản:} Đăng ký, đăng nhập, xác thực người dùng an toàn.
    \item \textbf{Chức năng nhắn tin:} Gửi nhận tin nhắn văn bản, hình ảnh và tệp tin theo thời gian thực; cập nhật trạng thái tin nhắn.
    \item \textbf{Chức năng gọi điện:} Thiết lập cuộc gọi thoại và gọi video cá nhân (1-1) với chất lượng ổn định.
    \item \textbf{Quản lý danh bạ:} Tìm kiếm người dùng, kết bạn và tạo nhóm trò chuyện cơ bản.
\end{itemize}

Sản phẩm được triển khai trên nền tảng Web, hướng tới sự tương thích tốt với các trình duyệt phổ biến hiện nay như Google Chrome, Microsoft Edge và Firefox.

\section{Định hướng giải pháp}
\label{section:1.3}
Để giải quyết các vấn đề đã nêu và đạt được mục tiêu đề ra, đồ án lựa chọn hướng tiếp cận phát triển ứng dụng Web dựa trên kiến trúc Client-Server kết hợp với mô hình Peer-to-Peer cho tính năng gọi điện. Cụ thể:

Về mặt công nghệ, hệ thống sẽ được xây dựng dựa trên nền tảng Node.js, một môi trường chạy JavaScript mạnh mẽ cho phía server, cho phép xử lý lượng lớn kết nối đồng thời với độ trễ thấp. Giao thức WebSocket (thông qua thư viện Socket.io) sẽ được sử dụng làm nòng cốt cho việc truyền tải tin nhắn tức thời và các tín hiệu điều khiển cuộc gọi (signaling).

Đối với tính năng gọi thoại và video, đồ án sử dụng công nghệ WebRTC. Đây là công nghệ nguồn mở cho phép các trình duyệt giao tiếp trực tiếp với nhau (Peer-to-Peer) để truyền tải luồng dữ liệu âm thanh và hình ảnh mà không cần thông qua server trung gian để chuyển tiếp dữ liệu media, giúp giảm tải cho server và tăng tốc độ truyền tin.

Phía người dùng (Frontend) sẽ được phát triển bằng thư viện ReactJS để đảm bảo giao diện trực quan, tương tác mượt mà và trải nghiệm người dùng tốt nhất. Cơ sở dữ liệu MongoDB sẽ được sử dụng để lưu trữ thông tin người dùng và lịch sử tin nhắn nhờ tính linh hoạt và khả năng mở rộng cao.

Đóng góp chính của đồ án là việc tích hợp thành công các công nghệ trên vào một sản phẩm thực tế, chứng minh khả năng làm chủ kỹ thuật lập trình mạng phức tạp và cung cấp một giải pháp thay thế khả thi cho các nhu cầu giao tiếp nội bộ hoặc quy mô nhỏ.

\section{Bố cục đồ án}
\label{section:1.4}
Phần còn lại của báo cáo đồ án tốt nghiệp được tổ chức như sau. Chương 2 tập trung khảo sát các hệ thống tương tự hiện có và phân tích chi tiết các yêu cầu chức năng, phi chức năng của hệ thống cần xây dựng. Chương 3 trình bày cơ sở lý thuyết về các công nghệ chủ chốt được sử dụng trong đề tài, bao gồm Node.js, WebSocket và giao thức WebRTC. Tiếp theo, Chương 4 đi sâu vào thiết kế kiến trúc hệ thống, thiết kế cơ sở dữ liệu, giao diện người dùng, đồng thời mô tả quá trình triển khai và đánh giá kết quả thực nghiệm. Chương 5 thảo luận về các giải pháp kỹ thuật cụ thể và những đóng góp nổi bật của đề tài trong việc tối ưu hóa hiệu năng hoặc trải nghiệm người dùng. Cuối cùng, Chương 6 tổng kết lại những kết quả đã đạt được, nhìn nhận các hạn chế còn tồn tại và đề xuất hướng phát triển tiếp theo cho hệ thống.

\end{document}